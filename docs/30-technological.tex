\section{Технологический раздел}

\subsection{Выбор языка и среды разработки}

Несмотря на то, что в ядро Linux с версии 6.1 добавлена поддержка языка программирования Rust \cite{Rust}, для написания драйвера был выбран язык программирования C, так как язык Rust внедряется в качестве эксперимента.

Для написания демона был выбран язык программирования С++ и библиотека Qt для создания и отображения виртуальной клавиатуры.

Для сборки обоих программ будет использована утилита Make. В качестве среды разработки был выбран редактор VSCode.

На момент написания данной работы новейшей версией ярда Linux была версия 6.1. LTS версия ядра -- 5.15.

Загружаемый модуль ядра будет написан для версий ядра 5.15 и 6.1. Для учета особенностей каждой из версий будет использована условная компиляция.

\subsection{Структуры драйвера}

Для хранения данных, связанных с подключенным устройством была объявлена структура, показанная на листинге \ref{lst:driver:struct}.

\begin{code}
	\captionof{listing}{Структура \texttt{usb\_joystick\_kbd}}
	\label{lst:driver:struct}
	\inputminted
	[
	frame=single,
	framerule=0.5pt,
	framesep=20pt,
	fontsize=\footnotesize,
	tabsize=4,
	linenos,
	numbersep=5pt,
	xleftmargin=10pt,
	]
	{text}
	{code/driver.h}
\end{code}

Для передачи события изменения позиции указателя на виртуальной клавиатуре объявлена структура, показанная на листинге \ref{lst:driver:event}.

\begin{code}
	\captionof{listing}{Структура \texttt{joystick\_event}}
	\label{lst:driver:event}
	\inputminted
	[
	frame=single,
	framerule=0.5pt,
	framesep=20pt,
	fontsize=\footnotesize,
	tabsize=4,
	linenos,
	numbersep=5pt,
	xleftmargin=10pt,
	]
	{text}
	{code/event.h}
\end{code}

\clearpage

\subsection{Точки входа драйвера}

\begin{code}
	\captionof{listing}{функция \texttt{probe} USB драйвера}
	\inputminted
	[
	frame=single,
	framerule=0.5pt,
	framesep=20pt,
	fontsize=\footnotesize,
	tabsize=4,
	linenos,
	numbersep=5pt,
	xleftmargin=10pt,
	]
	{text}
	{code/probe.c}
\end{code}

\begin{code}
	\captionof{listing}{функция \texttt{disconnect} USB драйвера}
	\inputminted
	[
	frame=single,
	framerule=0.5pt,
	framesep=20pt,
	fontsize=\footnotesize,
	tabsize=4,
	linenos,
	numbersep=5pt,
	xleftmargin=10pt,
	]
	{text}
	{code/disconnect.c}
\end{code}

%Для использования возможностей подсистемы ввода необходимо зарегистрировать драйвер устройства ввода. Создание структуры драйвера может быть выполнено вызовом одной из функции

%\begin{small}
%\begin{verbatim}
%struct input_dev *input_allocate_device(void);
%struct input_dev *devm_input_allocate_device(struct device *dev);
%\end{verbatim}
%\end{small}

%Вторая функция использует механизм управляемых ресурсов устройств \cite{devres}. Это позволяет сократить количество ошибок, связанных с очищением памяти после использования, в связи с применением счетчика ссылок. Для описания ресурсов устройства в ядре имеется специальная структура \texttt{devres}.

%\begin{small}
%\begin{verbatim}
%typedef void (*dr_release_t)(struct device *dev, void *res);
%
%struct devres_node {
%    struct list_head entry;
%    dr_release_t release;
%    const char *name;
%    size_t size;
%};
%
%struct devres {
%    struct devres_node node;
%    u8 __aligned(ARCH_KMALLOC_MINALIGN) data[];
%};
%\end{verbatim}
%\end{small}

%При удалении структуры устройства из системы, все связанные ресурсы будут освобождены посредством вызова функций \texttt{dr\_release\_t}.

\subsection{Функция обработки URB пакета}

\begin{code}
	\captionof{listing}{Функция обработки URB пакета}
	\label{lst:driver:urb-handle}
	\inputminted
	[
	frame=single,
	framerule=0.5pt,
	framesep=20pt,
	fontsize=\footnotesize,
	tabsize=4,
	linenos,
	numbersep=5pt,
	xleftmargin=10pt,
	]
	{text}
	{code/driver.c}
\end{code}

\subsection{Функция чтения файла событий}

\begin{code}
	\captionof{listing}{Функция чтения файла событий}
	\label{lst:driver:proc-read}
	\inputminted
	[
	frame=single,
	framerule=0.5pt,
	framesep=20pt,
	fontsize=\footnotesize,
	tabsize=4,
	linenos,
	numbersep=5pt,
	xleftmargin=10pt,
	]
	{text}
	{code/proc_event.c}  
\end{code}

\subsection{Точки входа подсистемы ввода для интерактивных устройств}

\begin{code}
	\captionof{listing}{Функции для работы с устройством ввода подсистемы \texttt{input}}
	\label{lst:driver:input-dev}
	\inputminted
	[
	frame=single,
	framerule=0.5pt,
	framesep=20pt,
	fontsize=\footnotesize,
	tabsize=4,
	linenos,
	numbersep=5pt,
	xleftmargin=10pt,
	]
	{text}
	{code/input_dev.c}  
\end{code}

\subsection{Файлы конфигурации сервисов}

Для запуска программы в виде демона с использованием утилиты systemd необходимо создать юнит-файл с конфигурацией сервиса. На листинге \ref{lst:service:virt-kbd} приведено содержимое файла \texttt{joystick\_virt\_kbd.service}.

\begin{code}
	\captionof{listing}{Конфигурационный файл \texttt{joystick\_virt\_kbd.service}}
	\label{lst:service:virt-kbd}
	\inputminted
	[
	frame=single,
	framerule=0.5pt,
	framesep=20pt,
	fontsize=\small,
	tabsize=4,
	linenos,
	numbersep=5pt,
	xleftmargin=10pt,
	]
	{text}
	{code/joystick_virt_kbd.service}  
\end{code}

На листинге \ref{lst:service:makefile} представлена часть файла сборки, отвечающая за установку и удаление сервиса из подсистемы \texttt{systemd}.

\begin{code}
	\captionof{listing}{makefile для сборки демона}
	\label{lst:service:makefile}
	\inputminted
	[
	frame=single,
	framerule=0.5pt,
	framesep=20pt,
	fontsize=\small,
	tabsize=4,
	linenos,
	numbersep=5pt,
	xleftmargin=10pt,
	]
	{text}
	{code/sysd_makefile}  
\end{code}

\subsection{Сборка и запуск}

Для сборки загружаемого модуля ядра достаточно выполнить команду \texttt{make} в папке с кодом модуля, после чего загрузка модуля в ядро осуществляется командой \texttt{insmod joystick\_kbd.ko}.

Сборка и запуск демона выполняется аналогичным образом. Выполнение команды \texttt{make} без параметров приведет к сборке исполняемого файла. Чтобы установить его в систему, необходимо выполнить команду \texttt{make install}. При этом сервис будет сразу же запущен. Для остановки и удаления программы из системы необходимо использовать команду \texttt{make uninstall}.

\pagebreak
