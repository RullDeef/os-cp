\section{Технологический раздел}

На момент написания данной работы новейшей версией ярда Linux была версия 6.1. LTS версия ядра -- 5.15.

Загружаемый модуль ядра будет написан для версий ядра 5.15 и 6.1. Для учета особенностей каждой из версий будет использована условная компиляция.

Демон будет написан в виде сервиса утилиты systemd.

\subsection{Выбор средств реализации}

Несмотря на то, что в ядро Linux с версии 6.1 добавлена поддержка языка программирования Rust, для написания драйвера был выбран язык программирования C, так как внедрение языка Rust является экспериментальной особенностью.

Для написания демона был выбран язык программирования С++. Также было решено использовать библиотеку Qt для создания и отображения виртуальной клавиатуры.

Для сборки обоих программ будет использована утилита Make.

В качестве среды разработки была выбран редактор VSCode.

\clearpage

\subsection{Реализация алгоритмов}

На листинге \ref{lst:driver:urb-handle} приведена реализация обработки корректного URB пакета.

\begin{code}
	\captionof{listing}{Реализация алгоритма обработки корректного блока URB}
	\label{lst:driver:urb-handle}
	\inputminted
	[
	frame=single,
	framerule=0.5pt,
	framesep=20pt,
	fontsize=\footnotesize,
	tabsize=4,
	linenos,
	numbersep=5pt,
	xleftmargin=10pt,
	]
	{text}
	{code/driver.c}
\end{code}

На листинге \ref{lst:driver:proc-read} приведена реализация функций работы с файлом в виртуальной файловой системе /proc.

\begin{code}
	\captionof{listing}{Реализация функций работы с файлом событий}
	\label{lst:driver:proc-read}
	\inputminted
	[
	frame=single,
	framerule=0.5pt,
	framesep=20pt,
	fontsize=\footnotesize,
	tabsize=4,
	linenos,
	numbersep=5pt,
	xleftmargin=10pt,
	]
	{text}
	{code/proc_event.c}  
\end{code}

На листинге \ref{lst:driver:input-dev} представлена реализация функций для работы с устройством ввода подсистемы \texttt{input}.

\begin{code}
	\captionof{listing}{Реализация функций для работы с устройством ввода подсистемы \texttt{input}}
	\label{lst:driver:input-dev}
	\inputminted
	[
	frame=single,
	framerule=0.5pt,
	framesep=20pt,
	fontsize=\footnotesize,
	tabsize=4,
	linenos,
	numbersep=5pt,
	xleftmargin=10pt,
	]
	{text}
	{code/input_dev.c}  
\end{code}

\subsection{Файлы конфигурации сервисов}

Для запуска программы в виде демона с использованием утилиты systemd необходимо создать юнит-файл с конфигурацией сервиса. На листинге \ref{lst:service:virt-kbd} приведено содержимое файла \texttt{joystick\_virt\_kbd.service}.

\begin{code}
	\captionof{listing}{Конфигурационный файл \texttt{joystick\_virt\_kbd.service}}
	\label{lst:service:virt-kbd}
	\inputminted
	[
	frame=single,
	framerule=0.5pt,
	framesep=20pt,
	fontsize=\small,
	tabsize=4,
	linenos,
	numbersep=5pt,
	xleftmargin=10pt,
	]
	{text}
	{code/joystick_virt_kbd.service}  
\end{code}

\clearpage

На листинге \ref{lst:service:makefile} представлена часть файла сборки, отвечающая за установку и удаление сервиса из подсистемы \texttt{systemd}.

\begin{code}
	\captionof{listing}{makefile для сборки демона}
	\label{lst:service:makefile}
	\inputminted
	[
	frame=single,
	framerule=0.5pt,
	framesep=20pt,
	fontsize=\small,
	tabsize=4,
	linenos,
	numbersep=5pt,
	xleftmargin=10pt,
	]
	{text}
	{code/sysd_makefile}  
\end{code}

\pagebreak
