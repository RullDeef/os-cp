\begin{center}
    {\bfseries\Large ЗАКЛЮЧЕНИЕ}
\end{center}
\addcontentsline{toc}{section}{ЗАКЛЮЧЕНИЕ}

В ходе работы был произведен анализ подсистемы USB и подсистемы ввода, рассмотрены основные моменты использования геймпада в качестве мыши и клавиатуры, решены ключевые проблемы с использованием функциональности, предоставляемой ядром операционной системы Linux.

Был разработан драйвер для геймпада, а также написана программа, запускаемая в режиме демона с использованием утилиты \texttt{systemd}.

Работоспособность и корректность выполнения были протестированы на реальном устройстве Logitech F310 для двух версий ядра 5.15 и 6.1.

Также было проведено исследование зависимости времени обработки URB от работы демона.

\pagebreak
