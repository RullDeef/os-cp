\section{Аналитический раздел}

\subsection{Драйвер устройства в ОС Linux}

Linux -- "модульная"\, операционная система. Для расширения функционала используются загружаемые модули ядра.

Прикладные программы не могут обращаться к устройствам напрямую. Вся работа с устройствами должна происходить с использованием средств, предоставляемых операционной системой. Реализация взаимодействия операционной системы с новым внешним устройством требует написания драйвера -- управляющей программы.

Драйверы устройств -- частный случай модуля ядра (actually -- модуль ядра в качестве своей составной части может включать в себя код драйвера устройства).

\subsection{Подсистема USB}

В настоящее время существует множество различных протоколов физического уровня (?), которые используются при подключении внешних устройств. Для устройств ввода, таких как мыши, клавиатуры и геймпады (Human Interaction Device (HID)) наибольшее распространение получил протокол USB.

В ядре операционной системы Linux имеется подсистема предназначенная для работы с USB-устройствами.

Так как целевое устройство -- геймпад -- подключается по USB, в дальнейшем будет использована именно эта подсистема. (не нужно писать драйвер с "чистого листа").

\subsubsection{Протокол взаимодействия геймпада Logitech F310}

Спецификацию работы стандартного драйвера xpad для большинства геймпадов можно прочитать здесь [https://docs.kernel.org/input/gamepad.html].

\subsection{Подсистемы ввода}

Существует множество типов устройств ввода. Примерами могут являться различные датчики температуры, давления, акселерометры, АЦП и ЦАП, и др.

В ядре Linux есть подсистема промышленного ввода-вывода (Industrial I/O) которая предназначена для поддержки устройств, которые в том или ином смысле производят аналого-цифровые и/или цифро-аналоговые преобразования. Данная подсистема была разработана чтобы заполнить пропасть между уже имеющимися подсистемами hwmon и input. Подсистема hwmon предназначена для датчиков с низкой частотой дискретизации, используемых для мониторинга и управления самой системой, таких как регулирование скорости вентилятора или измерение температуры. Подсистема input, как следует из ее названия, ориентирована на устройства ввода для взаимодействия с человеком (клавиатура, мышь, сенсорный экран). В некоторых случаях они значительно пересекаются с промышленным вводом-выводом.

//TODO: возможно, стоит рассмотреть Industrial I/O (iio) и процессы взаимодействия с ним.

Для реализации управления мышью через геймпад можно использовать джойстики (один для перемещения мыши, второй для управления колесиком). Кнопки геймпада A и B можно назначить на кнопки мыши (левую и правую соответственно).

\subsection{Виртуальная клавиатура}

Ввод символов с использованием геймпада -- нетривиальная задача. Одним из возможных вариантов является использование виртуальной клавиатуры и указателя, который можно перемещать по ней с помощью D-pad секции на геймпаде. Ввод символа под указателем можно осуществлять нажатием кнопки X на геймпаде. Таким образом можно вводить любой символ, располагающийся на виртуальной клавиатуре.

Однако для удобства пользователя нужно иметь возможность отобразить на экране виртуальную клавиатуру вместе с текущей позицией указателя.

Так как работа с дисплеем напрямую из ядра требует учета множества факторов (например таких как установленный оконный менеджер, состояние дисплея) управление отображением виртуальной клавиатуры становится трудоёмкой задачей.

Более оптимальным вариантом является написание демона, который по запросу будет выводить на экран виртуальную клавиатуру. Благодаря наличию большого числа графических библиотек уровня пользователя, отображение клавиатуры из демона является посильной задачей даже для непрофессионала.

\subsection{Взаимодействие драйвера и демона}

В операционной системе Linux существует множество способов передачи информации из ядра в пространство пользователя и наоборот.

-- сокеты семейства AF\_UNIX.

-- программные каналы.

-- ...

\subsection*{Выводы}

Для управления мышью и клавиатурой с использованием геймпада необходимо написать загружаемый модуль ядра с USB-драйвером, а также демона, предоставляющего сервис -- отображение виртуальной клавиатуры.

При написании драйвера будет использована подсистема ввода (input) так как она предназначена именно для ввода информации, поступающего от человека, а не от измерительных приборов и т.п.

Задача демона будет заключаться в прослушивании сокета(рили ???? может так и сделать?.. Или все же обычный файл в проце будет эффективнее, да) и отслеживании поступающих событий. По запросу он должен открывать окно с виртуальной клавиатурой и отображать перемещение виртуального указателя.

% Список:
% \begin{itemize}[leftmargin=1.6\parindent]
% 	\item[---] первое;
% 	\item[---] второе;
% 	\item[---] пятое;
% 	\item[---] десятое.
% \end{itemize}

\pagebreak
